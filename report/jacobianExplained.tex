Explanation for example in \texttt{howToComputeMultipleJacobiansAtOnce.m}.

The following function is an object, which represents the mathematical
\begin{align}
  x(x_0,q).
\end{align}
It maps $\mathbb{R}^2 \times \mathbb{R}^2 \mapsto \mathbb{R}^2$.
\begin{lstlisting}[basicstyle=\small]
  I = casadi.integrator('I','cvodes',ode_struct,opts);
\end{lstlisting}

The object 
\begin{lstlisting}[basicstyle=\small]
  I_fwd = I.factory('I_fwd',{'x0','p','fwd:x0','fwd:p'},{'fwd:xf'}) 
\end{lstlisting}
represents the mathematical function
\begin{align}
  f(x_0,q_0, d_{x_0},d_{q}) = G_x(x_0,q) d_{x_0} + G_p (x_0,q) d_q.
\end{align}
It maps $\mathbb{R}^2 \times \mathbb{R}^2 \times \mathbb{R}^2 \times \mathbb{R}^2\mapsto \mathbb{R}^2$.
If we want to compute the full Jacobian, we need to evaluate the function four times. 
For every direction once. 
Casadi allows to do this by one function call with
\begin{align}
  f(x_0,q_0, \begin{bmatrix} 1 0 0 0  \\ 0 1 0 0 \end{bmatrix},\begin{bmatrix}  0 0 1 0  \\  0 0 0 1 \end{bmatrix}).
\end{align}
This is equivalent to 
\begin{align}
  f(
    \begin{bmatrix}
    x_0  x_0  x_0  x_0
    \end{bmatrix},
    \begin{bmatrix}
    q_0   q_0 q_0 q_0
    \end{bmatrix},
  \begin{bmatrix} 1 0 0 0  \\ 0 1 0 0 \end{bmatrix}
  ,\begin{bmatrix}  0 0 1 0  \\  0 0 0 1 \end{bmatrix}).
\end{align}
because internally $x_0$ and $q_0$ gets duplicated and the function $f$ is evaluated column wise.
In order to compute the Jacobian for multiple values, we need to duplicate the initial values and the directions as well.
To evaluate the Jacobian for $x_0^1,x_0^2x_0^3$ and $q_0^1, q_0^2, q_0^3$ we have to do the function call 
\begin{align}
  f(
    &\begin{bmatrix}
    x_0^1  x_0^1  x_0^1  x_0^1
    x_0^2  x_0^2  x_0^2  x_0^2
    x_0^3  x_0^3  x_0^3  x_0^3
    \end{bmatrix},  \\
    &\begin{bmatrix}
    q_0^1   q_0^1 q_0^1 q_0^1
    q_0^2   q_0^2 q_0^2 q_0^2
    q_0^3   q_0^3 q_0^3 q_0^3
    \end{bmatrix}, \\
    &\begin{bmatrix} 1 0 0 0  1 0 0 0 1 0 0 0 \\ 0 1 0 0 0 1 0 0 0 1 0 0\end{bmatrix}, \\
    &\begin{bmatrix}  0 0 1 0  0 0 1 0 0 0 1 0  \\   0 0 0 1  0 0 0 1 0 0 0 1 \end{bmatrix}).
\end{align}
and split up the result in the individual Jacobians afterwards.