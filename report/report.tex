\documentclass[12pt]{article}


\usepackage{amsmath}
\usepackage{amssymb}
\usepackage{mathtools} 


\newcommand{\mtrx}[1]{\begin{bmatrix}#1\end{bmatrix}}

\begin{document}
 
Parametric optimal control problem:

\begin{subequations}
	\begin{alignat}{3} \label{eq:ocp}
	&\underset{x(\cdot), u(\cdot) }{\text{min}} \qquad \mathrlap{\Phi(x(t_f))}	\\
	&\qquad \text{s.t.}\qquad	&\dot{x}(t) 	&= f(x(t), u(t),p),  &  \label{eq:ivp1}	\\
	&				& x(t_0)	&= x_0						\label{eq:ivp2}		\\
	&				& x^{lo}	&\leq x(t) \leq x^{up},			\\
	&				& u^{lo}	&\leq u(t) \leq u^{up}	& \forall t 	& \in \left[t_0,t_f \right]
	\end{alignat}
\end{subequations}

Discretiezed with multiple shooting
variables: $s_0, \cdots, s_N$ $q_0, \cdots, q_N$


\begin{subequations}
	\begin{alignat}{3} \label{eq:ocp}
	&\underset{x(\cdot), u(\cdot) }{\text{min}} \qquad \mathrlap{\Phi(S_N)}	\\
	&\qquad \text{s.t.}\qquad	&  s_{i+1}	&= x(t_{i+1}; t_i, s_i,q_i,p)	& \ \ i &= 0,...,N-1		\\
	&				& s_0	    &= x_0							                                                      \\
	&				& x^{lo}	&\leq s_i \leq x^{up},	                          &     i &= 0, ..., N		\\
	&				& u^{lo}	&\leq q_i \leq u^{up}	                            &     i &= 0, ..., N
	\end{alignat}
\end{subequations}

$x(t; t_0, s,q,p)$ is the solution of 
\begin{align}
  \dot{x}(t) &= f(x(t),q,p) \\
  x(t_0) &= s
\end{align}

\begin{align}
  a(s,q) &=   \mtrx{   x_0 - s_0 \\
                      x(t_1;t_0,s_0,q_0,p) -s_1\\
                      \vdots    \\
                      x(t_N;t_{N-1},s_{N-1},q_{N-1},p) -s_N \\
  } \\
  b(s,q) &= \mtrx{ x^{lo} - s \\
                   s - x^{up} \\
                   q^{lo} - q \\
                   q- q^{up}}
\end{align}


Lagrange function at point $(s,q,\lambda, \mu)$

\begin{align}
  \mathcal{L}(s,q,\lambda, \mu) = \Phi(s_N) - \lambda ^\top a(s,q)-  \mu^\top b(s,q)
\end{align}

\begin{align}
  \nabla \mathcal{L}(s,q,\lambda, \mu) = 
  \mtrx{
            \nabla_x \Phi(s_n) -  \nabla_x a(s,q) \lambda - \nabla_x b(s,q) \mu  \\
            a(s,q)    \\
            b(s,q)
  }
\end{align}

\begin{align}
  \nabla^2 \mathcal{L}(s,q,\lambda, \mu) =
  \mtrx{
    \nabla^2 \Phi(s,q) - \nabla^2 a(s,q)\lambda  & \nabla a(s,q) & \nabla b(s,q)\\
    \nabla a(s,q) ^\top\\
    \nabla b(s,q)^\top
  }
\end{align}

At iteration $(x_i,\lambda_i, \mu_i)$ the following qp needs to be solved:
\begin{subequations}
	\begin{alignat}{3} \label{eq:ocp}
	&\underset{x(\cdot), u(\cdot) }{\text{min}} \qquad \mathrlap{\Delta x^\top \nabla^2_x \mathcal{L}(x_i,\lambda_i, \mu_i)\Delta x +\nabla \Phi(x_i)\Delta x}	\\
	&\qquad \text{s.t.}\qquad	& a(x_i) + \nabla a(x_i) \Delta x = 0 \\
	&				                  & b(x_i) + \nabla b(x_i) \Delta x \leq 0
	\end{alignat}
\end{subequations}

The following terms include the evaluation of the dynamical system: $a(x_i), \nabla_x a(x_i)\lambda, \nabla_x^2$

Functions for the following terms needs to be implemented:
$x(t;t,s,q,p)$, $\nabla_{(s,q)} x(t;t,s,q,p)\lambda$ $\nabla^2_{(s,q)} x(t;t,s,q,p) \lambda$



Dimensions:

\begin{align}
  t &\in \mathbb{R}                \\
  x &\in \mathbb{R}^{n_x}         \\
  q &\in \mathbb{R}^{n_q}         \\
  p &\in \mathbb{R}^{n_p}         \\
\end{align}

\begin{itemize}
    \item \texttt{inp.thoriz} Integration horizon in the form of a $2 \times 1$ matrix. At the beginning  this can be assumed to be [0,1]. Later time transformation.
    \item \texttt{inp.sd} Initial value for differential states 
    \item \texttt{inp.sa} Initial value for algebraic states -> ignore atm
    \item \texttt{inp.q } Control parameter
    \item \texttt{inp.p} Model parameter
    \item \texttt{inp.sensdirs} Matrix of directional derivates of dimension $1+n_x+n_q+n_p \times n_{sens}$, where $n_{sens}$ is the number of derivates.
         Order of directions: [t, x, q, p]. $t$ can be assumed to be 0.
    \item \texttt{inp.lambda} Adjoint directional derivatives. 
           Matrix of dimension $n_x \times n_{adj}$, where $n_{adj}$ is the number of adjoint derivates.
\end{itemize}


Output -> see readme


\textbf{Attention:} Implementation of $\nabla^2_{(s,q)} x(t;t,s,q,p) \lambda$ is the most challenging task. 
Here adjoint variables are needed. 
You have to think about the dimension of the results object. Make sure you initialize the integrator correctly. 
You should understand the context, why exactly these evaluations are needed.




\end{document}

